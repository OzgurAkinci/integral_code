\documentclass{article}
\usepackage{amsmath}
\usepackage{amssymb}
\usepackage{cancel}
\usepackage{setspace}
\usepackage{graphicx}
\usepackage{enumitem}
\usepackage[colorlinks=true, allcolors=blue]{hyperref}
\usepackage[english]{babel}
\usepackage[letterpaper,top=2cm,bottom=2cm,left=1cm,right=2cm,marginparwidth=1 cm]{geometry}
\usepackage{nicefrac, xfrac}
\usepackage{indentfirst}
\onehalfspacing
\begin{document}
\setlength\parindent{0pt}
Buradan katsayi cozumleri asagidaki gibi belirlenir.\\
\subsection{Alan Hesabi}
$P_{2}(x)$ polinomunun integrali alinir.\\
\[ I=\int_{0}^{2h} P_{2}(x) \,dx \]
\[ I=\int_{0}^{2h} (c_{0}+c_{1}x+c_{2}x^{2}) \,dx \]
\[ I=(c_{0}x+c_{1}\frac{x^{2}}{2}+c_{2}\frac{x^{3}}{3})\bigg\vert_{0}^{2h} \]
Burada $x$ yerine $2h$ yerlestirildiginde polinom ifadesinde bulunan $c_{k}*x^{k}$ terimlerinin tamam h ortak parantezine alinabilmektedir.\\
$\displaystyle I=h*((y_{0})*2+(-\frac{3}{2}y_{0}+2y_{1}-\frac{1}{2}y_{2})*\frac{2^{2}}{2}+(\frac{1}{2}y_{0}-y_{1}+\frac{1}{2}y_{2})*\frac{2^{3}}{3})$\\\\
$\displaystyle I=h*(\frac{1}{3}y_{0}+\frac{4}{3}y_{1}+\frac{1}{3}y_{2})$\\\\
$\displaystyle I=\frac{1}{3}*h*(y_{0}+4y_{1}+y_{2})$\\\\
h ve y degerleri ile integral hesaplanir.\\
$\displaystyle I=\frac{1}{3}*0.15707963267948966*(1*0.5877852522924731+4*0.7071067811865475+0.8090169943749475)$\\\\
$\displaystyle I=0.2212324925$\\\\
\end{document}
