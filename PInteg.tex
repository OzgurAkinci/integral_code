\documentclass{article}
\usepackage{amsmath}
\usepackage{amssymb}
\usepackage{cancel}
\usepackage{setspace}
\usepackage{graphicx}
\usepackage{enumitem}
\usepackage[colorlinks=true, allcolors=blue]{hyperref}
\usepackage[english]{babel}
\usepackage[letterpaper,top=2cm,bottom=2cm,left=1cm,right=2cm,marginparwidth=1 cm]{geometry}
\usepackage{nicefrac, xfrac}
\usepackage{indentfirst}
\onehalfspacing
\begin{document}
\setlength\parindent{0pt}
\section{Sayisal integral icin Polinom Yaklasim Formulleri}
\subsection{Polinom ifadesi}
Bir $P_{n}(x)$ polinomu genel olarak asagidaki ifade ile temsil edilir.\\
$\displaystyle P_{n}(x)= c_{0}+c_{1}x+c_{2}x^{2}+c_{3}x^{3}+\cdot\cdot\cdot+c_{n}x^{n}$
\subsection{Ornek Noktalar}
Bu ifade $x_{m}=x_{0}+mh$ biciminde secilen her bir nokta icin duzenlenir.\\
$\displaystyle P_{2}(0*h)=y_{0}=c_{0}+0*c_{1}*h+0^{2}*c_{2}*h^{2}$\\
$\displaystyle P_{2}(1*h)=y_{1}=c_{0}+c_{1}*h+c_{2}*h^{2}$\\
$\displaystyle P_{2}(2*h)=y_{2}=c_{0}+2*c_{1}*h+2^{2}*c_{2}*h^{2}$\\
\subsection{Denklem Sistemi}
Bu denklemlerden katsayi matrisi olusturulur.
\begin{center}
$$ \left[\begin{array}{rrr|r}
c_{0} & c_{1} & c_{2}\\
1 & 0 & 0^{2} & y_{0}\\
1 & 1 & 1^{2} & y_{1}\\
1 & 2 & 2^{2} & y_{2}\\
\end{array}\right] $$
\end{center}
Denklemler, uslu ifadeler hesaplanarak asagidaki gibi yeniden duzenlenir.
\begin{center}
$$ \left[\begin{array}{rrr|r}
c_{0} & c_{1} & c_{2}\\
1 & 0 & 0 & y_{0}\\
1 & 1 & 1 & y_{1}\\
1 & 2 & 4 & y_{2}\\
\end{array}\right] $$
\end{center}
\subsection{Denklem cozumu}
1. satir kullanilarak  asagisindaki (2,1) elemani 0 yapilir.\begin{center}
$$ \left[\begin{array}{rrr|r}
c_{0} & c_{1} & c_{2}\\
1 & 0 & 0 & y_{0}\\
0 & 1 & 1 & -y_{0}+y_{1}\\
1 & 2 & 4 & y_{2}\\
\end{array}\right] $$
\end{center}
1. satir kullanilarak  asagisindaki (3,1) elemani 0 yapilir.\begin{center}
$$ \left[\begin{array}{rrr|r}
c_{0} & c_{1} & c_{2}\\
1 & 0 & 0 & y_{0}\\
0 & 1 & 1 & -y_{0}+y_{1}\\
0 & 2 & 4 & -y_{0}+y_{2}\\
\end{array}\right] $$
\end{center}
2. satir kullanilarak  asagisindaki (3,2) elemani 0 yapilir.\begin{center}
$$ \left[\begin{array}{rrr|r}
c_{0} & c_{1} & c_{2}\\
1 & 0 & 0 & y_{0}\\
0 & 1 & 1 & -y_{0}+y_{1}\\
0 & 0 & 2 & y_{0}-2y_{1}+y_{2}\\
\end{array}\right] $$
\end{center}
\begin{center}
$$ \left[\begin{array}{rrr|r}
c_{0} & c_{1} & c_{2}\\
1 & 0 & 0 & y_{0}\\
0 & 1 & 0 & -\frac{3}{2}y_{0}+2y_{1}-\frac{1}{2}y_{2}\\
0 & 0 & 2 & y_{0}-2y_{1}+y_{2}\\
\end{array}\right] $$
\end{center}
\begin{center}
$$ \left[\begin{array}{rrr|r}
c_{0} & c_{1} & c_{2}\\
1 & 0 & 0 & y_{0}\\
0 & 1 & 0 & -\frac{3}{2}y_{0}+2y_{1}-\frac{1}{2}y_{2}\\
0 & 0 & 2 & y_{0}-2y_{1}+y_{2}\\
\end{array}\right] $$
\end{center}
\begin{center}
$$ \left[\begin{array}{rrr|r}
c_{0} & c_{1} & c_{2}\\
1 & 0 & 0 & y_{0}\\
0 & 1 & 0 & -\frac{3}{2}y_{0}+2y_{1}-\frac{1}{2}y_{2}\\
0 & 0 & 2 & y_{0}-2y_{1}+y_{2}\\
\end{array}\right] $$
\end{center}
Katsayi matrisi birim matrise donusturulur.
\begin{center}
$$ \left[\begin{array}{rrr|r}
c_{0} & c_{1} & c_{2}\\
1 & 0 & 0 & y_{0}\\
0 & 1 & 0 & -\frac{3}{2}y_{0}+2y_{1}-\frac{1}{2}y_{2}\\
0 & 0 & 1 & \frac{1}{2}y_{0}-y_{1}+\frac{1}{2}y_{2}\\
\end{array}\right] $$
\end{center}
Buradan katsayi cozumleri asagidaki gibi belirlenir.\\
$\displaystyle c_{0}=y_{0}$\\
$\displaystyle c_{1}=\frac{1}{h}(-\frac{3}{2}y_{0}+2y_{1}-\frac{1}{2}y_{2})$\\
$\displaystyle c_{2}=\frac{1}{h^{2}}(\frac{1}{2}y_{0}-y_{1}+\frac{1}{2}y_{2})$\\
\subsection{Alan Hesabi}
$P_{2}(x)$ polinomunun integrali alinir.\\
\[ I=\int_{0}^{2h} P_{2}(x) \,dx \]
\[ I=\int_{0}^{2h} (c_{0}+c_{1}x+c_{2}x^{2}) \,dx \]
\[ I=(c_{0}x+c_{1}\frac{x^{2}}{2}+c_{2}\frac{x^{3}}{3})\bigg\vert_{0}^{2h} \]
Burada $x$ yerine $2h$ yerlestirildiginde polinom ifadesinde bulunan $c_{k}*x^{k}$ terimlerinin tamam h ortak parantezine alinabilmektedir.\\
$\displaystyle I=c_{0}*(2h)+c_{1}*\frac{(2h)^{2}}{2}+c_{2}*\frac{(2h)^{3}}{3}$\\\\
$\displaystyle I=h*((y_{0})*2+(-\frac{3}{2}y_{0}+2y_{1}-\frac{1}{2}y_{2})*\frac{2^{2}}{2}+(\frac{1}{2}y_{0}-y_{1}+\frac{1}{2}y_{2})*\frac{2^{3}}{3})$\\\\
$\displaystyle I=h*(\frac{1}{3}y_{0}+\frac{4}{3}y_{1}+\frac{1}{3}y_{2})$\\\\
$\displaystyle I=\frac{1}{3}*h*(y_{0}+4y_{1}+y_{2})$\\\\
h ve y degerleri ile integral hesaplanir.\\
$\displaystyle I=\frac{1}{3}*0.15707963267*(1*0.58778525229+4*0.70710678118+0.80901699436)$\\\\
$\displaystyle I=0.2212324925$\\\\
\end{document}
