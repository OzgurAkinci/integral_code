\documentclass{article}
\usepackage{amsmath}
\usepackage{amssymb}
\usepackage{cancel}
\usepackage{setspace}
\usepackage{graphicx}
\usepackage{enumitem}
\usepackage[colorlinks=true, allcolors=blue]{hyperref}
\usepackage[english]{babel}
\usepackage[letterpaper,top=2cm,bottom=2cm,left=1cm,right=2cm,marginparwidth=1 cm]{geometry}
\usepackage{nicefrac, xfrac}
\usepackage{indentfirst}
\onehalfspacing
\begin{document}
\setlength\parindent{0pt}
\section{Sayisal integral icin Polinom Yaklasim Formulleri}
\subsection{Polinom ifadesi}
Bir $P_{n}(x)$ polinomu genel olarak asagidaki ifade ile temsil edilir.\\
$\displaystyle P_{n}(x)= c_{0}+c_{1}x+c_{2}x^{2}+c_{3}x^{3}+\cdot\cdot\cdot+c_{n}x^{n}$
\subsection{Ornek Noktalar}
Bu ifade $x_{m}=x_{0}+mh$ biciminde secilen her bir nokta icin duzenlenir.\\
$\displaystyle P_{3}(0*h)=y_{0}=c_{0}+0*c_{1}*h+0^{2}*c_{2}*h^{2}+0^{3}*c_{3}*h^{3}$\\
$\displaystyle P_{3}(1*h)=y_{1}=c_{0}+c_{1}*h+c_{2}*h^{2}+c_{3}*h^{3}$\\
$\displaystyle P_{3}(2*h)=y_{2}=c_{0}+2*c_{1}*h+2^{2}*c_{2}*h^{2}+2^{3}*c_{3}*h^{3}$\\
$\displaystyle P_{3}(3*h)=y_{3}=c_{0}+3*c_{1}*h+3^{2}*c_{2}*h^{2}+3^{3}*c_{3}*h^{3}$\\
\subsection{Denklem Sistemi}
Bu denklemlerden katsayi matrisi olusturulur.
\begin{center}
$$ \left[\begin{array}{rrrr|r}
c_{0} & c_{1} & c_{2} & c_{3}\\
1 & 0 & 0^{2} & 0^{3} & y_{0}\\
1 & 1 & 1^{2} & 1^{3} & y_{1}\\
1 & 2 & 2^{2} & 2^{3} & y_{2}\\
1 & 3 & 3^{2} & 3^{3} & y_{3}\\
\end{array}\right] $$
\end{center}
Denklemler, uslu ifadeler hesaplanarak asagidaki gibi yeniden duzenlenir.
\begin{center}
$$ \left[\begin{array}{rrrr|r}
c_{0} & c_{1} & c_{2} & c_{3}\\
1 & 0 & 0 & 0 & y_{0}\\
1 & 1 & 1 & 1 & y_{1}\\
1 & 2 & 4 & 8 & y_{2}\\
1 & 3 & 9 & 27 & y_{3}\\
\end{array}\right] $$
\end{center}
\subsection{Denklem cozumu}
1. satir kullanilarak  asagisindaki (2,1) elemani 0 yapilir.\begin{center}
$$ \left[\begin{array}{rrrr|r}
c_{0} & c_{1} & c_{2} & c_{3}\\
1 & 0 & 0 & 0 & y_{0}\\
0 & 1 & 1 & 1 & -y_{0}+y_{1}\\
1 & 2 & 4 & 8 & y_{2}\\
1 & 3 & 9 & 27 & y_{3}\\
\end{array}\right] $$
\end{center}
1. satir kullanilarak  asagisindaki (3,1) elemani 0 yapilir.\begin{center}
$$ \left[\begin{array}{rrrr|r}
c_{0} & c_{1} & c_{2} & c_{3}\\
1 & 0 & 0 & 0 & y_{0}\\
0 & 1 & 1 & 1 & -y_{0}+y_{1}\\
0 & 2 & 4 & 8 & -y_{0}+y_{2}\\
1 & 3 & 9 & 27 & y_{3}\\
\end{array}\right] $$
\end{center}
1. satir kullanilarak  asagisindaki (4,1) elemani 0 yapilir.\begin{center}
$$ \left[\begin{array}{rrrr|r}
c_{0} & c_{1} & c_{2} & c_{3}\\
1 & 0 & 0 & 0 & y_{0}\\
0 & 1 & 1 & 1 & -y_{0}+y_{1}\\
0 & 2 & 4 & 8 & -y_{0}+y_{2}\\
0 & 3 & 9 & 27 & -y_{0}+y_{3}\\
\end{array}\right] $$
\end{center}
2. satir kullanilarak  asagisindaki (3,2) elemani 0 yapilir.\begin{center}
$$ \left[\begin{array}{rrrr|r}
c_{0} & c_{1} & c_{2} & c_{3}\\
1 & 0 & 0 & 0 & y_{0}\\
0 & 1 & 1 & 1 & -y_{0}+y_{1}\\
0 & 0 & 2 & 6 & y_{0}-2y_{1}+y_{2}\\
0 & 3 & 9 & 27 & -y_{0}+y_{3}\\
\end{array}\right] $$
\end{center}
2. satir kullanilarak  asagisindaki (4,2) elemani 0 yapilir.\begin{center}
$$ \left[\begin{array}{rrrr|r}
c_{0} & c_{1} & c_{2} & c_{3}\\
1 & 0 & 0 & 0 & y_{0}\\
0 & 1 & 1 & 1 & -y_{0}+y_{1}\\
0 & 0 & 2 & 6 & y_{0}-2y_{1}+y_{2}\\
0 & 0 & 6 & 24 & 2y_{0}-3y_{1}+y_{3}\\
\end{array}\right] $$
\end{center}
3. satir kullanilarak  asagisindaki (4,3) elemani 0 yapilir.\begin{center}
$$ \left[\begin{array}{rrrr|r}
c_{0} & c_{1} & c_{2} & c_{3}\\
1 & 0 & 0 & 0 & y_{0}\\
0 & 1 & 1 & 1 & -y_{0}+y_{1}\\
0 & 0 & 2 & 6 & y_{0}-2y_{1}+y_{2}\\
0 & 0 & 0 & 6 & -y_{0}+3y_{1}-3y_{2}+y_{3}\\
\end{array}\right] $$
\end{center}
\begin{center}
$$ \left[\begin{array}{rrrr|r}
c_{0} & c_{1} & c_{2} & c_{3}\\
1 & 0 & 0 & 0 & y_{0}\\
0 & 1 & 1 & 1 & -y_{0}+y_{1}\\
0 & 0 & 2 & 0 & 2y_{0}-5y_{1}+4y_{2}-y_{3}\\
0 & 0 & 0 & 6 & -y_{0}+3y_{1}-3y_{2}+y_{3}\\
\end{array}\right] $$
\end{center}
\begin{center}
$$ \left[\begin{array}{rrrr|r}
c_{0} & c_{1} & c_{2} & c_{3}\\
1 & 0 & 0 & 0 & y_{0}\\
0 & 1 & 1 & 0 & -\frac{5}{6}y_{0}+\frac{1}{2}y_{1}+\frac{1}{2}y_{2}-\frac{1}{6}y_{3}\\
0 & 0 & 2 & 0 & 2y_{0}-5y_{1}+4y_{2}-y_{3}\\
0 & 0 & 0 & 6 & -y_{0}+3y_{1}-3y_{2}+y_{3}\\
\end{array}\right] $$
\end{center}
\begin{center}
$$ \left[\begin{array}{rrrr|r}
c_{0} & c_{1} & c_{2} & c_{3}\\
1 & 0 & 0 & 0 & y_{0}\\
0 & 1 & 1 & 0 & -\frac{5}{6}y_{0}+\frac{1}{2}y_{1}+\frac{1}{2}y_{2}-\frac{1}{6}y_{3}\\
0 & 0 & 2 & 0 & 2y_{0}-5y_{1}+4y_{2}-y_{3}\\
0 & 0 & 0 & 6 & -y_{0}+3y_{1}-3y_{2}+y_{3}\\
\end{array}\right] $$
\end{center}
\begin{center}
$$ \left[\begin{array}{rrrr|r}
c_{0} & c_{1} & c_{2} & c_{3}\\
1 & 0 & 0 & 0 & y_{0}\\
0 & 1 & 0 & 0 & -\frac{11}{6}y_{0}+3y_{1}-\frac{3}{2}y_{2}+\frac{1}{3}y_{3}\\
0 & 0 & 2 & 0 & 2y_{0}-5y_{1}+4y_{2}-y_{3}\\
0 & 0 & 0 & 6 & -y_{0}+3y_{1}-3y_{2}+y_{3}\\
\end{array}\right] $$
\end{center}
\begin{center}
$$ \left[\begin{array}{rrrr|r}
c_{0} & c_{1} & c_{2} & c_{3}\\
1 & 0 & 0 & 0 & y_{0}\\
0 & 1 & 0 & 0 & -\frac{11}{6}y_{0}+3y_{1}-\frac{3}{2}y_{2}+\frac{1}{3}y_{3}\\
0 & 0 & 2 & 0 & 2y_{0}-5y_{1}+4y_{2}-y_{3}\\
0 & 0 & 0 & 6 & -y_{0}+3y_{1}-3y_{2}+y_{3}\\
\end{array}\right] $$
\end{center}
\begin{center}
$$ \left[\begin{array}{rrrr|r}
c_{0} & c_{1} & c_{2} & c_{3}\\
1 & 0 & 0 & 0 & y_{0}\\
0 & 1 & 0 & 0 & -\frac{11}{6}y_{0}+3y_{1}-\frac{3}{2}y_{2}+\frac{1}{3}y_{3}\\
0 & 0 & 2 & 0 & 2y_{0}-5y_{1}+4y_{2}-y_{3}\\
0 & 0 & 0 & 6 & -y_{0}+3y_{1}-3y_{2}+y_{3}\\
\end{array}\right] $$
\end{center}
Katsayi matrisi birim matrise donusturulur.
\begin{center}
$$ \left[\begin{array}{rrrr|r}
c_{0} & c_{1} & c_{2} & c_{3}\\
1 & 0 & 0 & 0 & y_{0}\\
0 & 1 & 0 & 0 & -\frac{11}{6}y_{0}+3y_{1}-\frac{3}{2}y_{2}+\frac{1}{3}y_{3}\\
0 & 0 & 1 & 0 & y_{0}-\frac{5}{2}y_{1}+2y_{2}-\frac{1}{2}y_{3}\\
0 & 0 & 0 & 1 & -\frac{1}{6}y_{0}+\frac{1}{2}y_{1}-\frac{1}{2}y_{2}+\frac{1}{6}y_{3}\\
\end{array}\right] $$
\end{center}
Buradan katsayi cozumleri asagidaki gibi belirlenir.\\
$\displaystyle c_{0}=y_{0}$\\
$\displaystyle c_{1}=\frac{1}{h}(-\frac{11}{6}y_{0}+3y_{1}-\frac{3}{2}y_{2}+\frac{1}{3}y_{3})$\\
$\displaystyle c_{2}=\frac{1}{h^{2}}(y_{0}-\frac{5}{2}y_{1}+2y_{2}-\frac{1}{2}y_{3})$\\
$\displaystyle c_{3}=\frac{1}{h^{3}}(-\frac{1}{6}y_{0}+\frac{1}{2}y_{1}-\frac{1}{2}y_{2}+\frac{1}{6}y_{3})$\\
\subsection{Alan Hesabi}
$P_{3}(x)$ polinomunun integrali alinir.\\
\[ I=\int_{0}^{3h} P_{3}(x) \,dx \]
\[ I=\int_{0}^{3h} (c_{0}+c_{1}x+c_{2}x^{2}+c_{3}x^{3}) \,dx \]
\[ I=(c_{0}x+c_{1}\frac{x^{2}}{2}+c_{2}\frac{x^{3}}{3}+c_{3}\frac{x^{4}}{4})\bigg\vert_{0}^{3h} \]
Burada $x$ yerine $3h$ yerlestirildiginde polinom ifadesinde bulunan $c_{k}*x^{k}$ terimlerinin tamam h ortak parantezine alinabilmektedir.\\
$\displaystyle I=c_{0}*(3h)+c_{1}*\frac{(3h)^{2}}{2}+c_{2}*\frac{(3h)^{3}}{3}+c_{3}*\frac{(3h)^{4}}{4}$\\\\
$\displaystyle I=h*((y_{0})*3+(-\frac{11}{6}y_{0}+3y_{1}-\frac{3}{2}y_{2}+\frac{1}{3}y_{3})*\frac{3^{2}}{2}+(y_{0}-\frac{5}{2}y_{1}+2y_{2}-\frac{1}{2}y_{3})*\frac{3^{3}}{3}+(-\frac{1}{6}y_{0}+\frac{1}{2}y_{1}-\frac{1}{2}y_{2}+\frac{1}{6}y_{3})*\frac{3^{4}}{4})$\\\\
$\displaystyle I=h*(\frac{3}{8}y_{0}+\frac{9}{8}y_{1}+\frac{9}{8}y_{2}+\frac{3}{8}y_{3})$\\\\
$\displaystyle I=\frac{3}{8}*h*(y_{0}+3y_{1}+3y_{2}+y_{3})$\\\\
h ve y degerleri ile integral hesaplanir.\\
$\displaystyle I=\frac{3}{8}*0.01*(1*0.0+3*0.00999983333+3*0.01999866669+0.0299955002)$\\\\
$\displaystyle I=0.0004499662$\\\\
\end{document}
